% Determines the type of document and font size
\documentclass[12pt,a4paper]{article}
% Font control
\usepackage{mathptmx} % Times Roman Font
\usepackage{helvet} % Arial/Helvetica font
\renewcommand{\familydefault}{\sfdefault} % Makes serif text all Helvetica
% Set up the page margins
\usepackage[left=2.5cm, right=2.5cm, top=2.5cm]{geometry}
% Allow graphics
\usepackage{graphicx}
% put graphs in place
\usepackage[section]{placeins}
% caption font
\usepackage[font=small,labelfont=bf]{caption}
% float package: picture position
\usepackage{float}
% symbols
\usepackage{textcomp, gensymb}
% hyperref
\usepackage{hyperref}
% Add your report title here
\title{OCNG/ATMO 651 Final Project: Linear Inverse Model of Tropical Sea Surface Temperatures}
% Add your name here
\author{Jinjun Liu}
\date{\today}

% The start of the document
\begin{document}
% This adds the title page
\maketitle
\thispagestyle{empty}
% This adds the abstract
\begin{abstract}

In this project, we employ the linear inverse model (LIM) to predict sea surface tempratures (SSTs).

\end{abstract}

\tableofcontents

% Move to a new page and set the page numbering from here
\clearpage % moves to the next page
\pagenumbering{arabic}

\section{Introduction} % The start of a new section

Penland and Magorian proposed a linear inverse model (LIM) to predict sea surface tempratures (SSTs) from satellite observations \cite{Penland1993}. The LIM is a linear regression model that uses the satellite observations as predictors and the SSTs as the response variable. The LIM is a simple model that can be easily implemented and is computationally efficient. The LIM is also a useful tool for data assimilation. In this project, we employ the LIM to predict SSTs.

\section{Dataset and Method}\label{dataset-method}

The Python script that processes the data and generates the figures is available at \url{https://github.com/jinjunliu/atmo-651/blob/master/Final/ATMO651\_Final.ipynb}.

\section{Results}\label{results}

\section{Aknowledgements}\label{acknowledgements}

Thanks to Dr. Ping Chang for providing the datasets and the guidance.

\bibliographystyle{abbrv}
\bibliography{library}

\end{document}
